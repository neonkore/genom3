%
% Copyright (c) 2009-2010 LAAS/CNRS
% All rights reserved.
%
% Permission to use, copy, modify, and distribute this software for any purpose
% with or without   fee is hereby granted, provided   that the above  copyright
% notice and this permission notice appear in all copies.
%
% THE SOFTWARE IS PROVIDED "AS IS" AND THE AUTHOR DISCLAIMS ALL WARRANTIES WITH
% REGARD TO THIS  SOFTWARE INCLUDING ALL  IMPLIED WARRANTIES OF MERCHANTABILITY
% AND FITNESS. IN NO EVENT SHALL THE AUTHOR  BE LIABLE FOR ANY SPECIAL, DIRECT,
% INDIRECT, OR CONSEQUENTIAL DAMAGES OR  ANY DAMAGES WHATSOEVER RESULTING  FROM
% LOSS OF USE, DATA OR PROFITS, WHETHER IN AN ACTION OF CONTRACT, NEGLIGENCE OR
% OTHER TORTIOUS ACTION,   ARISING OUT OF OR IN    CONNECTION WITH THE USE   OR
% PERFORMANCE OF THIS SOFTWARE.
%
%                                             Anthony Mallet on Mon May 31 2010
%

This chapter describes the \GenoM{} Input File Format (dotgen) semantics and
gives the syntax for dotgen grammatical constructs.

\section{Overview}

The  \GenoM{} Input  File  Format (dotgen)  is  the language  used to  formally
describe a \GenoM{} component in terms  of services and data types it provides.
A  description written  in  dotgen  completely defines  the  interface and  the
internals of a component.

A    description    of   the    dotgen    preprocessing    is   presented    in
\xref{sec:iff:preproc}{section~\ref{sec:iff:preproc}}. The grammar is presented
in    \xref{sec:iff:preproc}{section~\ref{sec:iff:preproc}}    and   associated
semantics is described  in the rest of this chapter either  in place or through
references to other sub sections of this chapter.

A source  file containing a dotgen  component specification must  have a ".gen"
extension. The description of the dotgen grammar uses a syntax notation that is
similar        to       Extended        Backus-Naur        Format       (EBNF).
\xref{table:iff:ebnf}{Table~\ref{table:iff:ebnf}}  lists  the  symbols used  in
this format and their meaning.

\begin{table}
\caption{dotgen EBNF symbols}
\centering
\begin{tabular}{|cl|}
\doublehline
Symbol & Meaning\\
\hline
::=                     & Definition.\\
\verb#|#                & Alternation.\\
text                    & Nonterminals.\\
"text"                  & Terminals.\\
( \ldots )              & Grouping.\\
\{ \ldots \}            & Repetition: may occur zero or any number of times.\\
\verb#[#\ldots\verb#]#  & Option: may occur zero or one time.\\
\hline
\end{tabular}
\label{table:iff:ebnf}
\end{table}

\section{Preprocessing}
\label{sec:iff:preproc}

A dotgen specification consists of one or more files that are preprocessed. The
preprocessing performs file inclusion  and macro substitution and is controlled
by directives introduced by lines having  {\tt \#} as the first character other
than white space.  The preprocessing is done according  to the specification of
the  preprocessor   in  ISO/IEC  9899:1999  Standard  for   the  C  Programming
Language.  The preprocessor  that is  used  is the  one available  on the  host
system, as configured during the build of \GenoM{}. It is invoked as a separate
process.

Lines beginning with {\tt \#}  (also called "directives") communicate with this
preprocessor.  White  space may appear before  the {\tt \#}.   These lines have
their  own  syntax (the  C  preprocessor  syntax),  independent of  the  dotgen
language; they may appear anywhere and  have effects that last until the end of
the translation  unit.  The primary use  of the preprocessing  facilities is to
include   definitions   (especially  type   definitions)   from  other   dotgen
specifications.  Text  in files  included with a  {\tt \#include}  directive is
treated as if it appeared in the including file.

\section{Dotgen grammar}

\input{dotgen-rules}

\section{Dotgen specification}

   \input{dotgen-rule-spec}
   \input{dotgen-rule-statement}
   \input{dotgen-rule-idlstatement}
   \input{dotgen-rule-genomstatement}
